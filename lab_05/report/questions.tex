\chapter*{Контрольные вопросы}
\addcontentsline{toc}{chapter}{Контрольные вопросы}

\subsubsection{\textbf{1. Назовите преимущества и недостатки аппаратных ускорителей на ПЛИС по сравнению с CPU и графическими ускорителями?}}

Достоинствами данной системы являются:
\begin{itemize}
	\item низкая стоимость в сравнению с аппаратными ускорителями;
	\item большая частота эмуляции;
	\item компактность.
\end{itemize}

\subsubsection{\textbf{2. Назовите основные способы оптимизации циклических конструкций ЯВУ, реализуемых в виде аппаратных ускорителей?}}
Основными недостатками аппаратных эмуляторов на ПЛИС являются: 
\begin{itemize}
	\item необходимость перекомпиляции проекта и переконфигурации ПЛИС при любом исправлении содержимого проекта;
	\item наличие специализированного программного обеспечения для разделения модели микросхемы на части для загрузки в отдельные ПЛИС.
\end{itemize}



\subsubsection{\textbf{3. Назовите этапы работы программной части ускорителя в хост системе?}}
Способы оптимизаций циклов:
\begin{itemize}
	\item конвейерная обработка циклов;
	\item разворачивание циклов;
	\item потоковая обработка;
\end{itemize}


\subsubsection{\textbf{4. В чем заключается процесс отладки для вариантов сборки Emulation-SW, Emulation-HW и Hardware?}}

\begin{enumerate}
	\item Этап 1: Инициализируется среда OpenCL.
	\item Этап 2. Приложение создает три буфера, необходимых для обмена данными с ядром: два буфера для передачи исходных данных и один для вывода результата (память должна быть выделена с выравниванием 4 КБ). 
	\item Этап 3. Запуск задачи на исполнение.
	\item Этап 4: После завершения работы всех команд выходной буфер R\_buf содержит результаты работы ядра. 
\end{enumerate}


\subsubsection{\textbf{5. Какие инструменты и средства анализа результатов синтеза возможно использовать в Vitis HLS для оптимизации ускорителей?}}
\begin{itemize}
	\item Программная эмуляция (Emulation-SW) - код ядра компилируется для работы на CPU хост-системы. Этот вариант сборки служит для верификации совместного исполнения кода хост-системы и кода ядра, для выявления синтаксических ошибок, выполнения отладки на уровне исходного кода ядра, проверки поведения системы.
	\item Аппаратная эмуляция (Emulation-HW) - код ядра компилируется в аппаратную модель (RTL), которая запускается в специальном симуляторе на CPU. Этот вариант сборки и запуска занимает больше времени, но обеспечивает подробное и точное представление активности ядра. Данный вариант сборки полезен для тестирования функциональности ускорителя и получения начальных оценок производительности.
	\item Аппаратное обеспечение (Hardware) - код ядра компилируется в аппаратную модель (RTL), а затем реализуется на FPGA. В результате формируется двоичный файл xclbin, который будет работать на реальной FPGA.
\end{itemize}



Компилятор Xilinx Vitis v++ является одним из наиболее удачных проектов в этой области, и позволяет генерировать из описаний на языке C/C++ синтезируемые низкоуровневые RTL проекты, которые затем отображаются на структуру ПЛИС. 
