\chapter*{Контрольные вопросы}
\addcontentsline{toc}{chapter}{Контрольные вопросы}

\subsubsection{\textbf{1. Назовите причины расслоения оперативной памяти.}}

Скорость обработки данных в процессоре в несколько раз превышает скорость доступа к информации, которая размещена в оперативной памяти. Необходимо, чтобы итоговая скорость выполнения команды процессором как можно меньше зависела от скорости доступа к коду команды и к используемым в ней операндам из памяти. 

Расслоение ОЗУ — один из аппаратных путей решения проблемы дисбаланса в скорости доступа к данным, размещенным в оперативной памяти, и производительностью процессора.

Физически ОЗУ представимо в виде объединения k устройств, способных хранить одинаковое количество информации и способных взаимодействовать с процессором независимо друг от друга. При этом адресное пространство организовано таким образом, что подряд идущие адреса, или ячейки памяти, находятся в соседних устройствах (блоках) оперативной памяти. Тем самым расслоение памяти в идеале увеличивает скорость доступа в k раз, плюс буфер команд позволяет сократить обращения к ОЗУ.

Поэтому можно сказать, что ОЗУ без расслоения памяти — один контроллер на все банки, но ОЗУ с расслоением памяти — каждый банк обслуживает отдельный контроллер.

\subsubsection{\textbf{2. Как в современных процессорах реализована аппаратная предвыборка.}}

Аппаратная предвыборка происходит неявно, без участия программиста или компилятора. Кэш-контроллер анализирует, по каким адресам и в каком порядке программа обращается к оперативной памяти, и пытается предугадать, какие данные вскоре могут понадобиться программе, и осуществляет их автоматическую предвыборку в кэш-память.

Разумеется, по последовательности обращений программы в оперативную память в общем случае невозможно предсказать, какие данные понадобятся впоследствии.

Например, обработка массива: если кэш-контроллер обнаруживает, что оперативная память последовательно опрашивается с некоторым фиксированным шагом, то он делает предположение, что это происходит обработка массива, и начинает загружать следующие блоки данных заранее.

Аппаратная предвыборка данных в кэш-память не застрахована от ошибок. Если кэш-контроллер распознал, что это не обход массива, то данные, которые он загрузит в кэш-память, могут и не понадобиться. Однако хуже ошибки может быть то, что они могут «вытеснить» какие-либо полезные данные, находившиеся в кэш-памяти, и их придется загружать еще раз. 

\subsubsection{\textbf{3. Какая информация хранится в TLB.}}
TLB (Translation-Lookaside Buffer) представляет собой память с ассоциативной выборкой, которая содержит 20-тиразрядные базовые адреса 32-х страниц, то есть, старшие 20 разрядов физического адреса страницы. Каждый из базовых адресов имеет свой признак (тег). В качестве тега используются старшие 20 разрядов линейного адреса, то есть поля TABLE и PAGE.

Случай, когда базовый адрес страницы находится в TLB, называется КЭШ-попаданием. 

Помимо тега для каждого базового адреса страницы в TLB хранится дополнительная информация, позволяющая определить, какую страницу можно заменить на вновь вводимую. Поскольку TLB хранит адреса только 32 страниц объема 4 кб каждая, то микропроцессор может непосредственно формировать физические адреса для 128 Кб памяти.

\subsubsection{\textbf{4. Какой тип ассоциативной памяти используется в кэш-памяти второго уровня современных ЭВМ и почему.}}

В современных компьютерах применяют кэш-память второго уровня, которая находится между процессором и ОП и еще больше повышает производительность ЭВМ.

Кэш-память 2-го уровня, как правило, унифицирована, т. е. может содержать как команды, так и данные.

Если она встроена в ядро ЦП, то говорят о S-cache (Secondary Cache, вторичный кэш), в противном случае -- о B-cache (Backup Cache, резервный кэш). В современных серверных ЦП объем S-cache составляет от одного до нескольких мегабайт, a B-cache -- до 64 Мбайт. 

\subsubsection{\textbf{5. Приведите пример программной предвыборки.}}

При программной предвыборке программист или компилятор явно вставляет в программу команды предвыборки данных по тому или иному адресу в оперативной памяти. 

Программные предвыборки потребляют ресурсы в процессоре, и использование слишком многих предвыборок может ограничить их эффективность. 

Примеры таких предвыборок -- предвыборку данных в цикле для независимости от информации находящейся вне цикла и предвыборку в основных блоках, которые часто исполняются, но которые редко используют ее не зависимо от целей предвыборки.
